
\section{实验内容}

\textbf{实验背景}:本次实验的硬件系统采用 MPU 和 MCU 双平台设计,其中 MPU 用于 Linux 操作系统开发,MCU 用于 ARM 体系结构与裸机编程。本次实验会在 Linux 系统下进行。

\textbf{实验前置知识}:本次实验需要使用 C 语言来编写文件读取函数,并利用虚拟机交叉编译代码后,在开发平台上运行。因此需要提前掌握一定的 C 语言基础。

\textbf{实验目标}:



\section{实验部署}

程序的测试是在运行 Ubuntu 20.04 虚拟机的 Windows 主机上进行。Ubuntu 中需要装测试以及交叉编译的依赖,可以通过运行 \code{setup-ubuntu.sh} 安装依赖。我们没有使用提供的交叉编译链。

开发板上需要将 STM32MP157 芯片启动拨码设为 EMMC 启动方式,即 101 状态,并插好电源开机。

开发板与主机可以直接通过以太网线链接(见章节 \ref{sec:problem}),或通过以太网线链接到路由器。连到路由器的优点是可以通过外网访问,让所有组员可以在线上合作。开发板与主机之间的文件拷贝以及命令执行通过 Ubuntu 自带的 SCP 以及 SSH(没有使用 XShell 或 Xftp)(SSH 设置见章节 \ref{sec:compile})。若开放给公网建议在开发板上设置公钥认证(见章节 \ref{sec:problem})。

\section{实验过程}

\subsection{源代码}


Part 2 是在 Part 1 的 \code{waveheader.h}、\code{audioplayer.h}、\code{audioplayer.c}、\code{main.c} 的基础上开发的。之前的代码在 Part 1 已经解释了所以不再阐述,而新添加的代码如下:

\begin{itemize}
    \item \code{audioplayer.h}:在 \code{AudioPlayer} 结构体中添加了保存播放器状态的变量,包括目前的音量、播放线程、播放/暂停状态、目前的时间戳、播放速度、是否重复播放等等。由于 Part 2 对一些功能没有要求,所以许多变量目前没有起到作用。
    \item \code{audioplayer.c}:主要实现了以下函数:
    \begin{itemize}
        \item \code{ap_play}:开始播放。其中调用了 \code{alsa-lib} 的以 \code{snd_pcm} 开头的函数。播放可以在新线程中运行,以免阻塞 UI 线程。
        \item \code{ap_pause}:暂停播放,目前的实现没有保存时间戳,所以再次调用 \code{ap_play} 会从头开始播放。
        \item \code{ap_set_volume}:改变播放音量。其中调用了 \code{alsa-lib} 的以 \code{snd_mixer} 开头的函数。
    \end{itemize}
    \item \code{audioplayer_tui.c}:通过 \code{ap_tui} 函数实现了文本用户界面。通过输入命令可以打开 WAVE 文件、播放/暂停、改变音量、设置重复播放、输出 WAVE 文件以及播放信息、以及退出程序。
    \item \code{main.c}:调用 \code{ap_tui}。
\end{itemize}

\subsection{编译与运行}\label{sec:compile}

测试编译(\code{make debug})以及交叉编译(\code{make xc})的命令都在 \code{Makefile} 中定义。使用的交叉编译器是 \code{arm-linux-gnueabihf-gcc}。需要链接两个库:\code{libasound}(\code{})

将可执行文件拷贝到开发板上用 \code{make scp},需要保证 \code{SXX_KEY}、\code{SXX_PORT}、\code{SXX_HOST} 变量与实际情况一致。在主机上和在开发板上正常运行程序的命令分别为 \code{make drun} 和 \code{make xrun}。


\section{实验结果}

以下是


\newpage

\section{实验心得}\label{sec:problem}

在实验过程中遇到的问题及其解决方法如下:

\begin{itemize}
    \item 编译 Alsa:为了得到开发板的平台信息,需要将 \code{config.guess} 拷贝到开发板上并运行,得到:\code{armv7l-unknown-linux-gnueabihf}
    
\end{itemize}
