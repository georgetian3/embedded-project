\section{Part 1}


\subsection{实现}

\subsection{编译与运行}

为开发板的 ARM 处理器进行交叉编译需要 Ubuntu 的 \code{g++-arm-linux-gnueabihf} 包:

\begin{codeblock}
sudo apt install g++-arm-linux-gnueabihf
arm-linux-gnueabihf-g++ wave.cpp -static -o music
\end{codeblock}

主机与开发板通过以太网线沟通。

\newpage

\subsection{结果}



\begin{minted}[style=bw]{console}
root@myir:~# ./music-arm 
filename: 1 argument(s) expected. 0 provided.
Usage: music [--help] [--version] filename

Positional arguments:
  filename      Filename of the wave file

Optional arguments:
  -h, --help    shows help message and exits
  -v, --version prints version information and exits
root@myir:~# ./music-arm music-arm 
"music-arm" is not a valid wave file
root@myir:~# ./music-arm test.wav 
================================
test.wav
--------------------------------
RIFFChunk
    ID            RIFF
    Size          589860
    Format        WAVE
FormatChunk
    ID            fmt
    Size          16
    AudioFormat   1
    NumChannels   2
    SampleRate    44100
    ByteRate      176400
    BlockAlign    4
    BitsPerSample 16
DataChunk
    ID            data
    Size          589824
================================
\end{minted}


\subsection{问题及其解决方法}

\begin{itemize}
    \item 开发板的 DHCP 没有自动配置 IP 地址。
    
    手动在开发板以及主机上配置 IP。

    \item 在开发板上运行时出现错误:
    
    \code{./music-arm: /lib/libc.so.6: version `GLIBC_2.34' not found (required by ./music-arm)}
    
    由于主机的 glibc 的版本大于开发板上的 glibc 版本,所以在主机上交叉编译的可执行文件在开发板上找不到需要的库。使用 g++ 的 \code{-static} 编译参数进行静态链接。
\end{itemize}