\section{Part 1}


\subsection{实现}

\subsection{编译}

为开发板的 ARM 处理器进行交叉编译需要 Ubuntu 的 \code{g++-arm-linux-gnueabihf} 包:

\begin{codeblock}
sudo apt install g++-arm-linux-gnueabihf
arm-linux-gnueabihf-g++ wave.cpp -static -o music
\end{codeblock}

\subsection{运行}

主机与开发板通过以太网线沟通。

\subsection{结果}


\begin{codeblock}
RIFFChunk
    ID            RIFF
    Size          1073210
    Format        WAVE
FormatChunk
    ID            fmt 
    Size          16
    AudioFormat   1
    NumChannels   2
    SampleRate    8000
    ByteRate      32000
    BlockAlign    4
    BitsPerSample 16
DataChunk
    ID            data
    Size          1072948
\end{codeblock}


\subsection{问题及其解决方法}

\begin{itemize}
    \item 开发板的 DHCP 没有自动配置 IP 地址。
    
    手动在开发板以及主机上配置 IP。

    \item 在开发板上运行时出现错误:\code{.....glibc2.3}
    
    由于主机的 glibc 的版本大于开发板上的 glibc 版本,所以在主机上交叉编译的可执行文件在开发板上找不到需要的库。使用 g++ 的 \code{-static} 编译参数进行静态链接。
\end{itemize}