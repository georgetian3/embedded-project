\section{实验内容}

\textbf{实验背景}:本次实验的硬件系统采用 MPU 和 MCU 双平台设计,其中 MPU 用于 Linux 操作系统开发,MCU 用于 ARM 体系结构与裸机编程。本次实验会在 Linux 系统下进行。

\textbf{实验前置知识}:本次实验需要使用 C 语言来编写文件读取函数,并利用虚拟机交叉编译代码后,在开发平台上运行。因此需要提前掌握一定的 C 语言基础。

\textbf{实验目标}:

\begin{enumerate}
    \item 使用系统 I/O 函数读取 WAVE  音频文件,并将 WAVE 音频文件的参数输出到命令行。
    \item 将读取音频文件的参数写入开发平台上的文本文件中。注:文件操作需经过交叉编译并在开发平台上运行。
\end{enumerate}

\section{实验部署}

程序的测试是在运行 Ubuntu 20.04 虚拟机的 Windows 主机上进行。Ubuntu 中需要装测试以及交叉编译的依赖,可以通过运行 \code{setup-ubuntu.sh} 安装依赖。我们没有使用提供的交叉编译链。

开发板上需要将 STM32MP157 芯片启动拨码设为 EMMC 启动方式,即 101 状态,并插好电源开机。

开发板与主机可以直接通过以太网线链接(见章节 \ref{sec:problem}),或通过以太网线链接到路由器。连到路由器的优点是可以通过外网访问,让所有组员可以在线上合作。开发板与主机之间的文件拷贝以及命令执行通过 Ubuntu 自带的 SCP 以及 SSH(没有使用 XShell 或 Xftp)(SSH 设置见章节 \ref{sec:compile})。若开放给公网建议在开发板上设置公钥认证(见章节 \ref{sec:problem})。

\section{实验过程}

\subsection{源代码}

程序由以下四个源代码文件组成:

\begin{itemize}
    \item \code{waveheader.h}:定义了 Waveform Audio File Format (WAVE) 的头格式。参考的标准:
    
    \url{https://datatracker.ietf.org/doc/html/draft-ema-vpim-wav-00}。
    \item \code{audioplayer.h}:包含函数声明,以及定义了保存音频播放器的状态的结构体 \code{AudioPlayer}。
    \item \code{audioplayer.c}:实现了以下函数:
    \begin{itemize}
        \item \code{ap_open}:读入 WAVE 文件并将信息写入 \code{AudioPlayer}。
        \item \code{ap_get_header_string}:格式化 WAVE 文件的参数并写入字符串。
        \item \code{ap_print_header}:将 \code{ap_get_header_string} 产生的字符串输出到命令行。
        \item \code{ap_save_header}:将 \code{ap_get_header_string} 产生的字符串写入文件。
        \item \code{ap_close}:释放 \code{ap_open} 所占用的资源。
    \end{itemize}
    \item \code{main.c}:处理命令行参数,调用 \code{AudioPlayer} 相关的函数,输出信息以及错误。
\end{itemize}

\subsection{编译与运行}\label{sec:compile}

测试编译(\code{make debug})以及交叉编译(\code{make xc})的命令都在 \code{Makefile} 中定义。使用的交叉编译器是 \code{arm-linux-gnueabihf-gcc}。由于 Part 1 没有使用第三方库,所以不需要做额外的链接。

将可执行文件拷贝到开发板上用 \code{make scp},需要保证 \code{SXX_KEY}、\code{SXX_PORT}、\code{SXX_HOST} 变量与实际情况一致。在主机上和在开发板上正常运行程序的命令分别为 \code{make drun} 和 \code{make xrun}。

\code{Makefile} 中有些注释掉的命令是为了编译以及链接 Part 2 中的 ALSA 库,可以忽略。

\section{实验结果}



\begin{multicols}{2}
\setlength{\columnseprule}{0.4pt}
\begin{minted}[style=bw, fontsize=\small]{console}
root@myir:~# ./audioplayer-arm 
Usage: ./audioplayer <filename>
root@myir:~# ./audioplayer-arm does-not-exist
Cannot open file
root@myir:~# ./audioplayer-arm audioplayer-arm
Invalid wave
root@myir:~# ./audioplayer-arm test.wav
RIFF chunk
    riff_id      RIFF
    chunk_size   589860
    format       WAVE
Format chunk
    format_id    fmt 
    format_size  16
    audio_format 1
    channels     2
    sample_rate  44100
    byte_rate    176400
    block_align  4
    bit_depth    16
Data chunk
    data_id      data
    data_size    589824
Saved header to test.wav.txt
root@myir:~# cat test.wav.txt
RIFF chunk
    riff_id      RIFF
    chunk_size   589860
    format       WAVE
Format chunk
    format_id    fmt 
    format_size  16
    audio_format 1
    channels     2
    sample_rate  44100
    byte_rate    176400
    block_align  4
    bit_depth    16
Data chunk
    data_id      data
    data_size    589824
\end{minted}
\end{multicols}

从以上在开发板上运行的结果可见此程序有以下功能:
\begin{itemize}
    \item 提示程序使用方式
    \item 提示文件不可读
    \item 提示文件不是格式正确的 WAVE 文件
    \item 输出 WAVE 参数
    \item 将 WAVE 参数写入文件
\end{itemize}

\newpage

\section{实验心得}\label{sec:problem}

在实验过程中遇到的问题及其解决方法如下:

\begin{itemize}
    \item 当开发板直接通过以太网线连主机,没有自动配置 IP 地址。可以在开发板以及主机上手动配置 IP。在开发板上运行:\code{ifconfig eth0 192.168.2.2},并且在主机上如下设置:

    \begin{itemize}
        \item IPv4 address: \code{192.168.2.1}
        \item Gateway: \code{192.168.2.0}
        \item Subnet mask: \code{255.255.255.0}
    \end{itemize}

    当开发板连路由器,会有 DHCP 自动配置 IP 地址。

    \item 原本使用的虚拟机 Ubuntu 22.10 交叉编译的可执行文件在开发板上运行时出现错误:
    
    \code[fontsize=\small]{./audioplayer: /lib/libc.so.6: version `GLIBC_2.34' not found (required by ./audioplayer)}
    
    由于主机的 glibc 的版本大于开发板上的 glibc 版本,所以在主机上交叉编译的可执行文件在开发板上找不到需要的库。使用 gcc 的 \code{-static} 编译参数进行静态链接可以解决此问题,但是在预先调研如何链接 ALSA 的 \code{libasound} 库时遇到了静态链接产生的问题。因此决定使用动态链接并降低主机的 glibc 版本,即从 Ubuntu 22.10 切换到 Ubuntu 20.04。

    \item 开发板启动后,操作系统会默认开启 \code{mxapp2} 程序。在 Linux 系统上,通常输入 \keys{\ctrl+\Alt+Fn} 会切换到命令行,但是 \code{mxapp2} 似乎禁用了此功能。将 \code{/home/mxapp2} 重命名到任何其他名字,比如 \code{mxapp2.disabled} 可以禁止 \code{mxapp2} 的启动。开发板会直接进入 Weston 界面,可以访问 Weston 自带的终端模拟器,也可以使用 \keys{\ctrl+\Alt+Fn} 切换到纯文本命令行。这对 Part 1 没有影响因为 Part 1 仅仅需要 SSH 运行程序,但 Part 3 需要开发图形界面,所以预先解决了这个问题。
    
    \item 开发板上的 SSH 服务器软件是 Dropbear 而不是熟悉的 OpenSSH,则公钥认证设置不同。大概的设置步骤如下:
    \begin{enumerate}
        \item 使用 \href{https://linux.die.net/man/8/dropbearkey}{\code{dropbearkey}} 生成公密钥并存放在 \code{/etc/dropbear/dropbear_rsa_host_key}。
        \item 使用 \href{https://manpages.ubuntu.com/manpages/trusty/man1/dropbearconvert.1.html}{\code{dropbearconvert}} 将 Dropbear 格式的密钥转换成为 Ubuntu 使用的 OpenSSH 格式的密钥,并将密钥拷贝到 Ubuntu。
        \item 修改开发板上的 \code{/etc/default/dropbear} 文件,为了开启公钥认证内容应该是
        
        \code{DROPBEAR_EXTRA_ARGS=" -s"}。
        \item 重启开发板
    \end{enumerate}
    
\end{itemize}
