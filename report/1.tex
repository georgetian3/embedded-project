\section{实验内容}

实验背景:该实验在什么平台下进行的,比如 ARM A7,Windows;

实验前置知识:需要对哪些内容进行了解;

实验目标:要达到什么结果和指标,要对哪些结果进行展示和量化对比。

\section{实验部署}

设备,系统(主从机的系统),运行环境,软件 IDE 等等,可以标注型号和相关性能的就标注出来;

硬件相关的实验可以放一张连接好的实物图。


---------------------

主机与开发板通过以太网通讯。


\section{实验过程}

具体实验中的操作步骤和重要的代码(注意,不要把所有代码全贴上来,只对你觉得最关键的代码环节,或者自己有创新型的优化进行说明,也可以放伪代码);

操作步骤中可能出现的问题以及自己的解决方式(说的越详细越好,能够体现出你对实验,理论和系统底层的深入理解)。


\section{实验结果}

对测试集的描述,及用图片或者图表对测试结果进行描述。

有没有例外(不符合输出要求,或者达不到输出要求的样例),并分析为
什么会出现这样的结果。

分析实验在哪些方面还有改进的空间,如何提升效果,优化代码(比如
在嵌入式编程环境中,我们需要尽可能的优化代码执行文件的大小和执
行的速度,以及代码是否包含安全漏洞和可能存在的内存泄露等等)。

\section{实验心得}

对本次实验任务的评价,比如你在本次实验中学到了什么,实验对你的编程
能力有没有提升,实验的难度是否过大或者过于简单,可以向助教和老师提出相
关的意见等等。

\section{编译与运行}

为开发板的 ARM 处理器进行交叉编译需要 Ubuntu 的 \code{g++-arm-linux-gnueabihf} 包:

\begin{codeblock}
sudo apt install g++-arm-linux-gnueabihf
arm-linux-gnueabihf-g++ wave.cpp -static -o music
\end{codeblock}

主机与开发板通过以太网线沟通。

\newpage

\section{结果}



\begin{minted}[style=bw]{console}
root@myir:~# ./music-arm 
filename: 1 argument(s) expected. 0 provided.
Usage: music [--help] [--version] filename

Positional arguments:
  filename      Filename of the wave file

Optional arguments:
  -h, --help    shows help message and exits
  -v, --version prints version information and exits
root@myir:~# ./music-arm music-arm 
"music-arm" is not a valid wave file
root@myir:~# ./music-arm test.wav 
================================
test.wav
--------------------------------
RIFFChunk
    ID            RIFF
    Size          589860
    Format        WAVE
FormatChunk
    ID            fmt
    Size          16
    AudioFormat   1
    NumChannels   2
    SampleRate    44100
    ByteRate      176400
    BlockAlign    4
    BitsPerSample 16
DataChunk
    ID            data
    Size          589824
================================
\end{minted}


\section{问题及其解决方法}

\begin{itemize}
    \item 开发板的 DHCP 没有自动配置 IP 地址。
    
    手动在开发板以及主机上配置 IP。

    \item 在开发板上运行时出现错误:
    
    \code{./music-arm: /lib/libc.so.6: version `GLIBC_2.34' not found (required by ./music-arm)}
    
    由于主机的 glibc 的版本大于开发板上的 glibc 版本,所以在主机上交叉编译的可执行文件在开发板上找不到需要的库。使用 g++ 的 \code{-static} 编译参数进行静态链接。
\end{itemize}