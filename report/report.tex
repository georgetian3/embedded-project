\section{实验内容}

使用 STM32MP157 开发板、ST-Link、OLED 显示单元、温湿度传感器,功能拓展板、以及 STM32CubeIDE 实现一个温湿度测量仪。

\section{实验过程}

\subsection{插线和引脚部分}

通过RGB测试案例,小组搞清楚了 ST-Link、开发板、拓展板、电池的连线方式,同时也找准了连接拓展模块以及其对应的 STM32CubeIDE 中的引脚的方式。按照相同的方式,对 OLED 显示单元和温湿度传感器分别插线并设置引脚。先找到两个拓展模块对应的原理图,找到 OLED 的 \code{SCL} 和 \code{SDA} 对应的线,以及温湿度对应的 \code{DQ} 线。然后将它们连到功能拓展板对应的 \code{P3} (有 \code{SDA}、\code{SCL} 标识)和 \code{P5} (\code{D0})。再按照对应的拓展板 \code{P1} 接口上的标号,与开发板的 I/O 部分接口进行连接,完成硬件部分的连接。使用到的原理图部分如图 \ref{fig:oled}、\ref{fig:hdt11}、\ref{fig:io} 所示。随后在 STM32CubeIDE 中 的 Pinout \& Configuration 部分,进行引脚的设置。在开发板原理图中,它提供了对应的引脚号,如 \code{P3} 中编号 11 的接口对应的引脚号便是 \code{PB7},而小组将温湿度传感器接到的便是编号 11,所以直接在软件中设置 \code{PB7},设置为 \code{GPIO_INPUT}、\code{Cortex-M4} 状态。对于 OLED,虽然原理图上标识的是 \code{I2C6 SDA} 和 \code{I2C6 SCL} 对应的 \code{PZ6}、\code{PZ7},但小组一开始设置的时候没注意,设置成了 \code{I2C2}。后来发现也能正常使用便没有进行修改。时钟部分就完全按照作业文档进行设置的。然后用和 RGB 案例时相同的方式,进行生成代码。

\begin{figure}[h!]
    \centering
    \begin{minipage}{0.45\textwidth}
        \centering
        \includegraphics[width=\textwidth]{imgs/oled.pdf}
        \caption{OLED P1原理图}
        \label{fig:oled}
    \end{minipage}
    \begin{minipage}{0.45\textwidth}
        \centering
        \includegraphics[width=\textwidth]{imgs/oled.pdf}
        \caption{温湿度传感器 P1原理图}
        \label{fig:hdt11}
    \end{minipage}
\end{figure}

\newpage

\begin{figure}[h!]
    \centering
    \includegraphics[width=0.75\textwidth]{imgs/expand.pdf}
    \caption{开发板 I/O 部分原理图和功能拓展板原理图}
    \label{fig:io}
\end{figure}

\begin{figure}[h!]
    \centering
    \includegraphics[width=0.75\textwidth]{imgs/pic.png}
    \caption{最终硬件插线情况}
\end{figure}

\newpage

\subsection{代码部分}

代码部分可以分为三个部分:DHT11 函数部分,\code{main.c} 中的初始化部分,以及 \code{while(1)} 内的部分。

对于 DHT11 函数部分,因为看到最终的 \code{temp} 和 \code{humi} 引用参数都是 \code{uint16_t},本小组的思路是先正常读取整数部分,然后将整数部分整体向左移位占据高八位,将第八位变成小数部分。然后再 \code{while(1)} 对 16 位的 \code{temp} 和 \code{humi} 再进行数据处理。因此再 DHT11 函数部分,实现的就相对简单,只是再 \code{DHT11_Read_Data} 部分将 40bit 的数据分成 5 个 8 byte 分别进行读取。在读取小数位对应的数据前将整数位进行向左移位。\code{DHT11_Read_Byte} 函数就是调用 \code{DHT11_Read_Bit} 八次,每次将结果向左移一位。比较复杂的是 \code{DHT11_Read_Bit} 函数。需要通过查看时序图,来对一开始的两段信号进行忽略。数据 0 和数据 1 的区别在于它们的长短。因为数据 0 一般是 26-28\si{\micro\second} 的高电平,所以本小组的策略是将开始输出数据后的前30\si{\micro\second} 放过去,就看第 31\si{\micro\second} 的时候。倘若那个时候是高电平,则是 1,反之则为 0 。参考的时序图如下:

\begin{figure}[h!]
    \centering
    \includegraphics[width=0.8\textwidth]{imgs/wave.png}
    \caption{DHT11 传感器通讯时序图}
\end{figure}

对于 \code{main.c} 初始化部分,主要是对 delay、OLED、DHT11 分别进行初始化,然后再提前设置一些后续需要用到的变量。其中 delay 初始化调用的是 \code{HAL_User_Delay_init} 函数,参数传的是和时钟信号对应的209MHz。OLED 就调用 \code{init} 函数,DHT11 则调用 \code{DHT11_WHILE} 函数。
对于 \code{while(1)} 内的代码部分,因为每次调用 \code{DHT11_Read_Data} 前都需要调用 \code{DHT11_Check} 来检查 DHT11 的状态,所以需要有一个 \code{while} 循环在前面确保 DHT11 可以检测到。然后对于 OLED 显示的内容,本小组把显示的部分分成了 7 个部分:3 个英文字符串的显示(“temp:”、“humid:”、“lab1:”),3 个数字部分(温度整数、温度小数、湿度整数),以及一个字符“.”。在这里需要对我们之前 \code{DHT11_Read_Data} 函数留下的两个 \code{uint16_t} 整数 \code{temp} 和 \code{humi} 进行处理。它们的高八位就是整数部分,所以在显示整数前要把两个都右移 8 位。对于小数部分则是用 \code{mod 256} 的方式得出。最后在每次显示后需要 \code{delay} 一段时间,再重新进行调用,我们设置的是 300ms,为了在测试以及录视频的时候更快的看到变化。当然,也不能忘了清空 OLED 来避免一些问题以及调用 \code{DHT11_Rst} 函数来进行重置。

\subsection{测试部分}

OLED 可以显示,调用的 \code{OLED_ShowString} 和 \code{OLED_ShowInt32Num} 正确无误。温湿度传感器可以接收到正确的数据(手放上去温度慢慢上升、湿度快速上升,手拿开数据会慢慢下降直至稳定状态)。演示视频有两个,一个是有 \code{OLED_Clear} 的版本(即最终提交的版本),因为涉及到刷新,所以会一闪一闪的,不是很好拍照。另一个是没有 \code{OLED_Clear} 的版本。

\section{实验结果}

OLED 可以显示,调用的 \code{OLED_ShowString} 和 \code{OLED_ShowInt32Num} 正确无误。温湿度传感器可以接收到正确的数据(手放上去温度慢慢上升、湿度快速上升,手拿开数据会慢慢下降直至稳定状态)。

\begin{figure}[h!]
    \centering
    \includegraphics[width=\textwidth]{imgs/res.png}
    \caption{实验结果}
\end{figure}

演示视频有两个,一个是有 \code{OLED_Clear} 的版本(即最终提交的版本),因为涉及到刷新,所以会一闪一闪的,不是很好拍照。另一个是没有 \code{OLED_Clear} 的版本。

\newpage

\section{过程中遇到的问题及解决方案}

本次实验中,小组遇到的问题较多,有些简单的问题在于对设备不是很熟悉,在查阅了课程讨论群中的 FAQ、实验文档、原理图之后便快速解决。下面列出的是一些在实验过程遇到的比较困难的问题。

\begin{enumerate}
    \item Debug 时卡在 \code{SystemClock_Config},运行的时候却没有任何问题。
    
    参考了下面链接中所提供的解决方案:\url{https://blog.csdn.net/tuxinbang1989/article/details/100826820},将 \code{SYS} 中的 \code{Timebase source} 设为 \code{Systick} 后即可。同时每次运行后想要 debug 也需要对开发板进行重启。

    \item 初始化 delay 时,\code{HAL_User_Delay_init} 不知道怎么设参数
    
    查看了往年的 FAQ,然后看了群内其他同学对这个函数的讨论,最终设置了和时钟设置相同的 209\si{\mega\hertz}。

    \item \code{DHT11_Check} 在 \code{while(1)} 里面不通过,但是在初始化阶段的 \code{DHT11_WHILE} 却可以正常通过。
    
    这个问题非常奇怪,我们求助了助教。助教让我们请教了其他组同学的代码,然后发现在 \code{main} 这两段代码的部分唯一的区别就只是我们试图在 \code{DHT11_WHILE} 初始化后对 OLED 进行 \code{show} 操作,想对屏幕显示进行一个初始化。可能是函数互相影响的问题,以至于在 \code{DHT11_WHILE} 初始化 DHT11 后,到 \code{DHT11_Check} 之间最好没有任何其他拓展模块函数的调用。我们最终去掉了屏幕初始化的部分,选择将 DHT11 传感器初始化、\code{DHT11_Check}、读取后,再统一对 OLED 进行操作。

    \item DHT11 通过 \code{DHT11_Check} 后只输出 1
    
    尝试了换个传感器,重新配置引脚,结果都没解决。最后发现是读写顺序和调用 \code{DHT11_Rst} 的问题。

    \item 湿度理应是 $<90$ 的,但手放上去后,会比 90 多一点。

    我们之前设置了对湿度数据的判定,即若大于 90 则表示 \code{DHT11_Read_Data} 失败,从而导致传感器湿度数据大于 90 时会黑屏(因为没有数据了)。后来去掉了这个限制,发现湿度确实会稍微超过 90 一些,大概 92、93。这个应该是正常现象。别的时候传感器也很正常。

\end{enumerate}


\section{心得体会}

本次实验感觉经历了三个不同的阶段:刚发下实验工具箱的时候的兴奋、开始编写代码测试出问题却找不到问题根源时的迷茫、以及最终做完实验后的成就感。硬件总是能在一开始让人诞生浓浓的兴趣,毕竟可以除了编写代码,可以捣鼓一堆小玩意儿。实验过程中就会有点抓瞎。实验本身涉及到的知识非常多,并且总有一些预料之外的事情会发生,这些都是调试阶段抓瞎的原因。光确保插线和引脚配置没问题就确认了好几次,每次出现预料之外的问题都会觉得会不会是硬件哪里我们配置错了。对于软件部分,我们也是尽可能用OLED来作为一个显式输出,来查看DHT11函数返回的值,进行调试。总的来说,在本次实验中,本小组对嵌入式系统有了进一步的认识与了解,对设备进行了实际的操作,收获颇丰。